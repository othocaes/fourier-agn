\documentclass[11pt,letterpaper]{article}

\usepackage{natbib}
\usepackage{graphicx}
\usepackage[margin=1.in,centering]{geometry}

\begin{document}

Consider two lightcurves $x(t)$ and $y(t)$, where $x(t)$ is the driving lightcurve and $y(t)$ is the reprocessed lightcurve.  If they are related by a linear impulse response, $g(\tau)$, then: 

\begin{equation}
y(t) = \int_{-\infty}^{\infty} g(\tau) x(t-\tau)  {\rm d}\tau
\end{equation} 

So, $y(t)$ is a delayed and blurred version of $x(t)$, with the amount of delay and blurring encoded in $g(\tau)$.

The power spectral density (PSD) of $x(t)$ is calculated from the Fourier transform of $x(t)$, which we denote $X(\nu)$.  The PSD is $|X(\nu)|^2 = X^*(\nu)X(\nu)$, where the $^*$ denotes the complex conjugate.  From the convolution theorem of Fourier transforms we can write:

\begin{equation}
Y(\nu) = G(\nu) X(\nu)
\end{equation}

This means it is easy to relate the PSD of the reprocessed lightcurve to the PSD of the driving lightcurve and the impulse response function:

\begin{equation}
|Y(\nu)|^2 = |G(\nu)|^2 |X(\nu)|^2
\end{equation}

The cross spectrum is defined as
\begin{equation}
C(\nu) = X^*(\nu) Y(\nu)
\end{equation}
the phase, $\phi$, of which gives the phase lag between X and Y at each Fourier frequency, $\nu$.  This can be converted to a time lag through: 
\begin{equation}
\tau(\nu) = \frac{\phi(\nu)}{2\pi\nu}
\end{equation}
Since $Y(\nu) = G(\nu) X(\nu)$,  the cross spectrum can be written as:
\begin{equation}
C(\nu) = X^*(\nu) G(\nu) X(\nu) =  G(\nu) |X(\nu)|^2 
\end{equation}
thus, for a given impulse response function, one can trivially predict the time lags as a function of frequency, $\tau(\nu)$, by calculating the phase of $G(\nu)$, and the frequency dependence of the lags directly relates to the shape of the response function.

\end{document}