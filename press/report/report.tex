% mnras_template.tex
%
% LaTeX template for creating an MNRAS paper
%
% v3.0 released 14 May 2015
% (version numbers match those of mnras.cls)
%
% Copyright (C) Royal Astronomical Society 2015
% Authors:
% Keith T. Smith (Royal Astronomical Society)

% Change log
%
% v3.0 May 2015
%    Renamed to match the new package name
%    Version number matches mnras.cls
%    A few minor tweaks to wording
% v1.0 September 2013
%    Beta testing only - never publicly released
%    First version: a simple (ish) template for creating an MNRAS paper

%%%%%%%%%%%%%%%%%%%%%%%%%%%%%%%%%%%%%%%%%%%%%%%%%%
% Basic setup. Most papers should leave these options alone.
\documentclass[a4paper,fleqn,usenatbib]{article}

% MNRAS is set in Times font. If you don't have this installed (most LaTeX
% installations will be fine) or prefer the old Computer Modern fonts, comment
% out the following line
\usepackage{newtxtext,newtxmath}
% Depending on your LaTeX fonts installation, you might get better results
% with one of these:
%\usepackage{mathptmx}
%\usepackage{txfonts}

% Use vector fonts, so it zooms properly in on-screen viewing software
% Don't change these lines unless you know what you are doing
\usepackage[T1]{fontenc}
\usepackage{ae,aecompl}


%%%%% AUTHORS - PLACE YOUR OWN PACKAGES HERE %%%%%

% Only include extra packages if you really need them. Common packages are:
\usepackage{graphicx}	% Including figure files
\usepackage{amsmath}	% Advanced maths commands
\usepackage{amssymb}	% Extra maths symbols

\begin{document}
%%%%%%%%%%%%%%%%%%%%%%%%%%%%%%%%%%%%%%%%%%%%%%%%%%

%%%%% AUTHORS - PLACE YOUR OWN COMMANDS HERE %%%%%

% Please keep new commands to a minimum, and use \newcommand not \def to avoid
% overwriting existing commands. Example:
%\newcommand{\pcm}{\,cm$^{-2}$} % per cm-squared

%%%%%%%%%%%%%%%%%%%%%%%%%%%%%%%%%%%%%%%%%%%%%%%%%%

%%%%%%%%%%%%%%%%%%% TITLE PAGE %%%%%%%%%%%%%%%%%%%

% Title of the paper, and the short title which is used in the headers.
% Keep the title short and informative.
\title[Optical Reverberation with Max Likelihood; NGC 5548]{Optical
Reverberation Mapping of NGC 5548 with Maximum Likelihood}

% The list of authors, and the short list which is used in the headers.
% If you need two or more lines of authors, add an extra line using \newauthor
\author[Ulrich et al.]{
Otho A. Ulrich,$^{2}$\thanks{E-mail: otho.a.ulrich@wmich.edu}
Edward M. Cackett,$^{1}$
\\
% List of institutions
$^{1}$Department of Physics and Astronomy, Wayne State University, 666 W.
Hancock St., Detroit, MI 48201, USA\\
$^{2}$Department of Physics, Western Michigan University, Kalamazoo, MI
49008-5252, USA\\
}

% These dates will be filled out by the publisher
%\date{Accepted XXX. Received YYY; in original form ZZZ}
\date{August 8, 2016}

% Enter the current year, for the copyright statements etc.
% \pubyear{2016}

% Don't change these lines
\label{firstpage}
%\pagerange{\pageref{firstpage}--\pageref{lastpage}}
\maketitle

% Abstract of the paper
\begin{abstract}
Power spectral densities and time delays of 19 wavelength bands are recovered
as part of a reverberation mapping of NGC 5548. The latest time-variable light
curves are made available in STORM III by \cite{2016ApJ...821...56F}. The
uneven distribution of flux data in those curves necessitates the use of a
maximum likelihood method in conjunction with Fourier transformations to
produce the frequency-dependent values of interest. Variability in the
emissions is confirmed in the power spectral densities, and the time delays
show the expected frequency dependence. The time delays also appear to have
wavelength dependence. There are issues computing accurate error estimates for
both distributions that remain as yet unresolved. The transfer function should
be recoverable once those and any additional computational issues are
resolved.
\end{abstract}

% Select between one and six entries from the list of approved keywords.
% Don't make up new ones.
\begin{keywords}
keyword1 -- keyword2 -- keyword3
\end{keywords}

%%%%%%%%%%%%%%%%%%%%%%%%%%%%%%%%%%%%%%%%%%%%%%%%%%

%%%%%%%%%%%%%%%%% BODY OF PAPER %%%%%%%%%%%%%%%%%%

\section{Introduction}
The local Type-I Seyfert galaxy NGC 5548, while perhaps the best-studied
active galaxy, remains an object of intense interest and study to modern
astronomy. An extensive observational campaign has been carried out on this
object, producing the most complete set of time-dependent light curves yet
collected from an active galactic nucleus (AGN). The physics underlying the
nature of these light curves is not completely understood, and so remains a
topic of debate and great interest.

	\subsection{Reverberation Mapping}
	A primary model of AGN suggests that an accretion disk is incident upon a
	central super-massive black hole (SMBH). Electromagnetic emission emergent
	from the accreting gases close to the SMBH is reprocessed by the
	surrounding gas clouds, resulting in observed response delays between
	emission peaks that are dependent on the geometry of the system. The
	impulse response encodes this geometry, and astronomers have combined
	models for the orbiting gas velocities and ionization states with these
	observed time delays to calculate it for some known systems. This
	technique has become a standard for calculating the black hole mass of
	AGN, and is well-described by \cite{2007MNRAS.380..669C} and
	\cite{2014A&ARv..22...72U}. It continues to be refined, and may also
	become a tool to measure the black hole spin of these systems
	\citep{2016arXiv160606736K}.

	(Probably would be good to put a picture here describing simple
	reverberation.)

	Many reverberation mapping techniques involve time-domain analyses, such
	as cross-correlation. Time-domain techniques have limitations:
	cross-correlation, for instance, provides only the average time delay
	between two light curves; they also require data that is evenly-sampled
	across the time domain. X-ray reverberation mapping in particular has
	developed a body of techniques based on frequency-domain techniques,
	primarily Fourier analysis; these techniques still require evenly-sampled
	data and have been enabled by the relatively good data coverage in X-ray
	bands. They provide the astronomer with more detailed information about
	the variability and response delay within the system compared to
	time-domain techniques.

	The power spectral density (PSD) as a function of temporal frequency for a
	given emission band can be produced using Fourier transforms, providing a
	measure of the time-scale of variability in that band. Given two bands,
	typically a reference or "driving" band and a delayed or "response" band,
	a cross spectrum can also be constructed. From the complex
	argument of the cross-correlation function, one can derive the
	frequency-dependent time delay between those bands; an important step
	toward
	constituting the transfer function of a system. Very good explanations of
	these techniques and the associated mathematics are available from
	\cite{2014A&ARv..22...72U}.

	A top-hat function provides a simple model of the impulse response of a
	delayed light curve. A fast Fourier transform method of this impulse
	response provides the time delay spectrum as a function of temporal
	frequency. This simple model provides a guideline for how the computed
	time delays are expected to be distributed as a function of temporal
	frequency.


	(Side-by-side graphic of top-hat impulse response function and FFT of
	top-hat giving time delays.)

	\subsection{Unevenly-Spaced Data}
 	
 	Some X-ray datasets contain gaps due to orbital mechanics, which motivated
	the work in \cite{2013ApJ...777...24Z}, where a maximum likelihood method
	is used to perform Fourier analysis on light curves with gaps. Since its
	development, this technique has found success among studies of
	observations captured by low-orbit X-ray telescopes that exceed the
	telescopes' orbital periods, such as the analysis performed by
	\cite{2016arXiv160606736K}. Until now, reverberation mapping in the
	optical bands has been limited to time-domain techniques. Many datasets
	available for these bands have uneven sampling across the time domain,
	however, and so do not lend themselves well to time-domain or traditional
	frequency-domain analyses. The maximum likelihood method is well-suited
	to extracting useful information from the data available in those
	datasets.



\section{Analysis}
The 1367\AA$ $ light curve, obtained from observations made with the Hubble
Space Telescope, is chosen as the reference curve. The power spectral
densities and time delays as a function of temporal frequency are computed for
each band in the dataset -- 18 bands not including the reference band.

The light curves analysed here are unevenly distributed along the time axis,
which suggests that the maximum likelihood method developed by
\cite{2013ApJ...777...24Z} is a reasonable candidate for producing the PSD and
time delays in the frequency domain. The latest version (CHECK THIS) of the
C++ program psdlag associated with that work is used to directly produce the
PSD and cross spectra. The time delay spectrum is produced from the cross
spectrum by dividing it by $2 \pi f$, with $f$ the mean frequency for a given
bin.

	\subsection{Dataset}
	\cite{2016ApJ...821...56F} published the best dynamic data yet collected
	from NGC 5548 over a 200-day (CHECK THIS) period, for 19 bands throughout
	the optical and into the UV spectra. These data were collected from a
	variety of observatories, including both space and ground-based
	telescopes, and thus have significantly variable sampling rates.

	(Include picture of Fausnaugh data here)

	\subsection{Error Analysis}
	For the presented set of resultant data, the error estimates are extracted
	from the covariance matrix. This method assumes that the errors between
	frequency bins are not correlated, so these values only represent a lower
	limit of the true variability. Scanning the likelihood function can
	provide better error estimates at the cost of computation time, as can
	running Monte Carlo simulations. All of these methods are built into the
	psdlag program provided by \cite{2013ApJ...777...24Z}, however, some
	issues have prevented proper error analysis using the latter two methods.
	This is discussed in more detail in section \ref{results}.

\section{Results}
\label{results}

	An atlas of the power spectral densities as functions
	of temporal	frequency for all 18 delayed bands is provided in this section.
	One is also provided of the time delay spectra for each band. The reference
	band PSD is also provided separately. Errors presented in these atlases
	are obtained from the covariance matrix.

	(Atlas of PSD)

	(Atlas of Time Delays)

	\subsection{Dubious Error Computations}
	The errors obtained from the covariance matrix are only a lower estimate
	of the true error. An error analysis by scanning the likelihood function
	was attempted, but dubious values led to their exclusion from these results.
	In the case of

	(Example of bad LF error)

	Monte Carlo simulations were also attempted as a way of estimating the
	variability of the resultant values. Many errors obtained from this method
	were much larger than the expected accurate values. Therefore, this analysis
	was also excluded.

	(Example of bad MC error)


\section{Discussion}
Frequency-dependent power spectral densities confirm time-dependent variability
in the emission strengths for each band. This behaviour is expected for
any active galactic nucleus and has been long-confirmed in NGC 5548, so
it comes as no surprise to find those results here. 

Analysis of the top-hat impulse response model predicted frequency-dependent
time delays, which have been recovered from the light curves in this analysis.
Furthermore, the distribution of time delays indicates a wavelength-dependent
nature. This warrants further study and analysis.

(Maybe a graph comparing the top-hat time delays to one band's time delays.)

The analyses performed on these data have elucidated clear trends in the PSD
and time delays. With reverberation mapping, the goal is to recover the transfer
function, which encodes the geometry of the system. Recovering the time delays
is a significant step toward that goal. 



%%%%%%%%%%%%%%%%%%%%%%%%%%%%%%%%%%%%%%%%%%%%%%%%%%

%%%%%%%%%%%%%%%%%%%% REFERENCES %%%%%%%%%%%%%%%%%%

% The best way to enter references is to use BibTeX:

\bibliographystyle{mnras}
\bibliography{wsu_reu} % if your bibtex file is called example.bib


% Alternatively you could enter them by hand, like this:
% This method is tedious and prone to error if you have lots of references
% \begin{thebibliography}{99}
% \bibitem[\protect\citeauthoryear{Author}{2012}]{Author2012}
% Author A.~N., 2013, Journal of Improbable Astronomy, 1, 1
% \bibitem[\protect\citeauthoryear{Others}{2013}]{Others2013}
% Others S., 2012, Journal of Interesting Stuff, 17, 198
% \end{thebibliography}

%%%%%%%%%%%%%%%%%%%%%%%%%%%%%%%%%%%%%%%%%%%%%%%%%%

%%%%%%%%%%%%%%%%% APPENDICES %%%%%%%%%%%%%%%%%%%%%

% \appendix
% 
% \section{Some extra material}
% 
% If you want to present additional material which would interrupt the flow of
% the main paper, it can be placed in an Appendix which appears after the list
% of references.

%%%%%%%%%%%%%%%%%%%%%%%%%%%%%%%%%%%%%%%%%%%%%%%%%%


% Don't change these lines
\bsp	% typesetting comment
\label{lastpage}
\end{document}

% End of mnras_template.tex